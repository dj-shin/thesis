\section{Approach}

To address the limitations identified in the previous section we use deep learning with different representation which contains more comprehensive semantics.
Applying deep learning for software analysis has advantages in many perspectives \cite{shin2015recognizing}.

To use deep learning the way of representing dataset, i.e. softwares, is required and should meet the following criteria:

% we need better ~ 대신 'DL을 쓸거고 새로운 representation을 쓸거다'
% 왜 DL을 쓰는지, 왜 이런 representation을 써야 하는지 

\textbf{Program Structure.} Detailed information of software should be included in order to achieve function-level granularity.
Representation should be able to determine in which function the vulnerability resides, which cannot be accomplished with simple code metrics such as LoC.
Therefore program structure should be included at some level to help model determine the location of vulnerability.
Both the raw form such as binary \cite{shin2015recognizing, kosmidis2017machine} and preprocessed form such as abstract syntax trees (AST) \cite{wang2016automatically}
have been used as representation of software for deep learning models.

% LoC 같은걸론 파악할 수 없다는 걸 말하기 위함인데 부가설명이 필요한듯
% dawn song 논문에 DL을 써야하는 이유
% => DL을 쓰는데 representation에 이러이런게 필요하다 => CPG

\textbf{Program Context.} Control flow and dependence of program are also required. Since same code can be either safe or vulnerable depending on the context.
Figure shows how almost identical programs can be classified differently by the context.
This difference cannot be distinguished by program structures only. Hence, additional information in representation is required.

% CPG 자체에 대한 언급은 아래 subsection으로. 조건만

\subsection{Representation}
We use code property graph which was introduced to effectively mine large amounts of source code for vulnerabilities \cite{yamaguchi2014modeling}.
Code property graph combines AST, control flow graph (CFG) and program dependency graph (PDG), therefore comprehensively includes requirements mentioned above.

\begin{itemize}
\item
AST
\item
CFG
\item
PDG
\end{itemize}

\subsection{Neural Network for Graph Data}

Since we will use code property graph as program representation, we need learning model which takes graph data as input.

\begin{itemize}
\item
Graph kernel

\item
Graph Neural Network

\item
Convolutional Neural Network

\begin{itemize}

\item
Convolutional neural network is typical deep learning model used widely in image recognition. Typically convolutional neural network takes 2-dimensional data, usually image, as input and feed forward them in network with convolution operations.
\end{itemize}

\end{itemize}